\begin{abstract}
    \setstretch{1.5}
    Trong bối cảnh nhu cầu truy xuất thông tin ngày càng tăng cao và đa dạng hóa về mặt dữ liệu, báo cáo này tiến hành nghiên cứu, triển khai và so sánh hai phương pháp truy xuất thông tin: Vector Space Model truyền thống và Neural Network-based Model hiện đại, áp dụng trên hai tập dữ liệu khác nhau: Cranfield và dữ liệu tuyển sinh của Trường Đại học Công nghệ Thông tin TP.HCM (UIT).

    Báo cáo hệ thống hoá kiến thức cơ bản về truy xuất thông tin, triển khai đúng trình tự các giai đoạn trong một hệ thống IR: tiền xử lý, lập chỉ mục, truy vấn, tính tương đống và xếp hạng. Đối với Vector Space Model, báo cáo phân tích các kĩ thuật biên trọng số, chỉ mục từ khoá và so khấu truy xuất dựa trên cosine similarity. Với Neural Network-based Model, báo cáo triển khai mô hình BERT để tạo ra các nhấn vector ngữ nghĩa cho cả truy vấn và tài liệu, sau đó lưu trữ trong chỉ mục vector và tiến hành truy xuất theo cosine similarity.

    Phân tích thực nghiên được tiến hành với cài đặt thực tế, đánh giá hiệu suất cả hai phương pháp dựa trên các độ đo chuẩn trong IR. Ngoài ra, báo cáo cũng để xuất và triển khai mô hình truy xuất hỗn hợp (Hybrid Search) áp dụng cho dữ liệu tuyển sinh UIT, kết hợp BM25 và BERT embedding, hướng đến việc xây dựng một hệ thống tra cứu trên ngôn ngữ tự nhiên cho sinh viên và phụ huynh. Báo cáo được thực hiện dưới sự hướng dẫn của thầy Nguyễn Trọng Chỉnh.
\end{abstract}