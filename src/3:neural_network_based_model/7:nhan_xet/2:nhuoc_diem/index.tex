\subsection{Nhược điểm}
Mặc dù mô hình Neural Network-based mang lại những cải tiến vượt trội trong khả năng truy xuất thông tin, song đi kèm với đó là một loạt hạn chế mang tính hệ thống cần được nhìn nhận một cách nghiêm túc. Trước hết, bản chất biểu diễn ngữ nghĩa thông qua vector khiến cho các chỉ mục do mô hình tạo ra thiếu tính minh bạch: không có cách rõ ràng nào để truy vết hoặc lý giải một thành phần cụ thể trong vector dẫn đến việc tài liệu được chọn, từ đó gây khó khăn trong việc diễn giải kết quả truy vấn. Điều này đặc biệt bất lợi trong những lĩnh vực đòi hỏi khả năng giải thích cao như y học, pháp lý hay tài chính, nơi mọi quyết định đều cần đi kèm với lập luận rõ ràng và có thể kiểm chứng. Bên cạnh đó, quá trình cập nhật chỉ mục cũng là một điểm nghẽn lớn: khi dữ liệu đầu vào thay đổi ở quy mô đáng kể, hệ thống thường phải tái lập toàn bộ không gian nhúng -- một quy trình vừa tốn kém thời gian, vừa đòi hỏi tài nguyên đáng kể. Hơn nữa, chi phí vận hành tổng thể của hệ thống Neural IR không hề nhỏ: từ giai đoạn huấn luyện mô hình cho đến lưu trữ các vector có chiều cao và thực hiện truy vấn theo thời gian thực, tất cả đều yêu cầu hạ tầng tính toán mạnh mẽ và tối ưu hóa kỹ lưỡng. Những hạn chế này cho thấy rằng việc ứng dụng các mô hình học sâu vào bài toán truy xuất thông tin đòi hỏi không chỉ năng lực kỹ thuật, mà còn cả sự cân nhắc chiến lược về chi phí, mục tiêu và khả năng duy trì dài hạn của hệ thống.

