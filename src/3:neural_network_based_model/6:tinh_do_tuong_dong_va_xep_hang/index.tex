\section{Tính độ tương đồng và xếp hạng}
Mặc dù khởi nguồn từ hai hướng tiếp cận khác biệt, mô hình Neural Network-based và mô hình Vector Space lại gặp nhau tại điểm hội tụ bản chất: cả hai đều biểu diễn tài liệu và truy vấn như những vector trong không gian nhiều chiều, từ đó mở ra cùng một cánh cửa cho việc đo lường tương đồng và xếp hạng dựa trên hình học vector. Trong mô hình Neural IR, các vector không còn được hình thành từ những trọng số thủ công như TF-IDF, mà được học một cách ngữ nghĩa và tự động thông qua các mạng nơ-ron sâu. Tuy nhiên, một khi các vector đã được thiết lập, phép đo tương đồng -- chẳng hạn như cosine similarity -- vẫn là nhịp cầu chính kết nối truy vấn và tài liệu, cũng giống như trong mô hình không học sâu. Do đó, phương pháp tính toán khoảng cách, góc giữa vector, hay độ liên quan, về mặt toán học không thay đổi, dù cách hình thành vector có thể khác biệt sâu sắc. Vì toàn bộ các khái niệm về tính tương đồng, xếp hạng và hình học vector đã được trình bày đầy đủ trong phần nói về mô hình Vector Space, phần hiện tại sẽ không lặp lại những lý thuyết này, mà thay vào đó tập trung vào những đặc điểm riêng có của việc sử dụng mạng nơ-ron để học biểu diễn.
