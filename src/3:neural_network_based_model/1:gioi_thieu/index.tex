\section{Giới thiệu}
Sự xuất hiện của các mô hình mạng nơ-ron trong lĩnh vực truy xuất thông tin (IR) đã khơi mở một chương mới đầy triển vọng, được ví như làn sóng thứ ba trong tiến trình phát triển của học máy, tiếp nối chuỗi thành tựu vang dội trong nhận dạng giọng nói, thị giác máy và xử lý ngôn ngữ tự nhiên. Trong bối cảnh đó, \textit{Neural IR} -- tức việc ứng dụng trực tiếp các kiến trúc mạng nơ-ron nông hoặc sâu vào các bài toán IR -- đã nhanh chóng khẳng định vai trò trung tâm trong việc kiến tạo nên những mô hình có khả năng học biểu diễn dữ liệu từ văn bản thô một cách hiệu quả, vượt xa giới hạn của các phương pháp dựa trên đặc trưng thủ công truyền thống.

Động lực cho sự trỗi dậy của Neural IR đến từ một tập hợp hài hòa giữa các yếu tố then chốt: tiến bộ trong thiết kế kiến trúc mạng nơ-ron như CNNs, RNNs và đặc biệt là Transformer; khả năng tiếp cận các tập dữ liệu lớn được gán nhãn đầy đủ -- yếu tố thiết yếu cho sự học hiệu quả của các mô hình sâu; cùng với sức mạnh tính toán ngày càng gia tăng, đặc biệt từ các GPU chuyên dụng. Điểm nổi bật của các mô hình nơ-ron hiện đại là khả năng chấp nhận đầu vào là toàn văn truy vấn và tài liệu, từ đó tự học các biểu diễn phù hợp cho việc xếp hạng hoặc truy vấn mà không cần thiết kế thủ công các đặc trưng như trước.

Ban đầu, các mô hình này thường được sử dụng như lớp xếp hạng lại trong cấu hình nhiều tầng, xử lý N tài liệu ứng viên được truy xuất bởi các phương pháp truyền thống. Tuy nhiên, phạm vi ứng dụng đã dần mở rộng sang các ngữ cảnh truy xuất đa dạng hơn như tìm kiếm ngữ nghĩa, gợi ý nội dung, tìm kiếm đa phương tiện và thậm chí là các hệ thống hội thoại sinh phản hồi tự nhiên. Dẫu vậy, điểm chung xuyên suốt là năng lực biểu diễn ngữ nghĩa sâu sắc từ văn bản -- một năng lực không thể đạt được nếu chỉ dừng lại ở các chỉ số thống kê bề mặt.

Bằng cách tận dụng sức mạnh biểu diễn của các mô hình mạng nơ-ron hiện đại, nghiên cứu này lựa chọn sử dụng kiến trúc \textit{BERT} làm nền tảng chính cho các thí nghiệm truy xuất thông tin phía sau -- như một minh chứng sống động cho tiềm năng mà Neural IR mang lại trong việc tái định nghĩa cách con người truy tìm tri thức trong kho tàng văn bản khổng lồ.
