\section{Truy xuất chỉ mục}
Trong hệ thống truy hồi thông tin hiện đại sử dụng mô hình mạng nơ-ron, quá trình truy xuất chỉ mục không còn đơn thuần là việc ánh xạ giữa từ khóa và tài liệu, mà trở thành một chuỗi các thao tác ngữ nghĩa sâu sắc được định hình bởi các biểu diễn phân bố. Từ thời điểm truy vấn được phát sinh, chuỗi xử lý bắt đầu với việc biểu diễn ngữ nghĩa của truy vấn người dùng bằng cách sử dụng một encoder mạng nơ-ron, thường là BERT hoặc các mô hình Transformer huấn luyện trước. Truy vấn, vốn có thể ngắn gọn nhưng giàu thông tin ngữ cảnh, được ánh xạ vào một vector trong không gian ngữ nghĩa chiều cao, nơi khoảng cách giữa các vector phản ánh mức độ tương quan ngữ nghĩa giữa các thực thể.

Trong khi đó, quá trình lập chỉ mục đã được thực hiện từ trước, nơi mỗi tài liệu hoặc đoạn văn bản được mã hóa thành vector ngữ nghĩa và được lưu trữ trong một cơ sở dữ liệu vector. Hệ thống lưu trữ này, có thể là FAISS hoặc các giải pháp tương tự, hỗ trợ tìm kiếm gần đúng theo cosine similarity hoặc khoảng cách Euclidean. Khi một truy vấn đã được mã hóa, hệ thống thực hiện phép so khớp giữa vector truy vấn và các vector tài liệu bằng cách tìm top-k tài liệu gần nhất trong không gian biểu diễn.

Giai đoạn truy xuất này tạo ra một tập các tài liệu ứng viên, thường được gọi là bước đầu tiên trong hệ thống hai giai đoạn (two-stage retrieval). Tuy nhiên, để đảm bảo chất lượng và độ chính xác của kết quả cuối cùng, hệ thống tiếp tục bước xếp hạng lại (re-ranking). Trong bước này, các mô hình mạng nơ-ron dạng cross-encoder, vốn có khả năng xử lý song song và đối chiếu từng cặp truy vấn - tài liệu, sẽ đánh giá độ phù hợp chi tiết giữa nội dung truy vấn và từng tài liệu ứng viên. Cơ chế attention trong các mô hình này cho phép phát hiện các điểm liên kết sâu giữa các thành phần ngôn ngữ, từ đó đưa ra điểm số mức độ liên quan ngữ nghĩa với độ chính xác cao.

Cuối cùng, danh sách các tài liệu được sắp xếp theo mức độ liên quan và trả về cho người dùng, thường cùng với các đoạn trích nổi bật (snippet) được trích xuất từ nội dung tài liệu để minh họa lý do tại sao tài liệu đó phù hợp với truy vấn. Nhờ sự kết hợp chặt chẽ giữa biểu diễn ngữ nghĩa và cơ chế học sâu, toàn bộ quá trình truy xuất chỉ mục trong hệ thống IR hiện đại không chỉ thể hiện độ chính xác cao, mà còn mở ra khả năng tìm kiếm thông minh, thích ứng với các truy vấn phức tạp, mơ hồ và đa nghĩa -- những thách thức vốn dĩ rất khó xử lý đối với các hệ thống truy hồi truyền thống
