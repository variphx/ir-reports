\subsection{Trọng số (weight) cho các vectors}
Trong không gian truy xuất thông tin, \textbf{việc xác định và gán trọng số (weights) cho các thành phần trong vector biểu diễn tài liệu và truy vấn} không chỉ đơn thuần là một thao tác kỹ thuật, mà còn là yếu tố cốt lõi, chi phối sâu sắc đến \textbf{độ chính xác, độ nhạy} và \textbf{khả năng phản ánh ngữ nghĩa thực tế} của hệ thống xếp hạng tài liệu. Thực tiễn cho thấy, không phải mọi từ trong ngôn ngữ đều sở hữu \textbf{mức độ quan trọng như nhau}, và do đó, việc sử dụng một cách cơ học \textbf{tần suất xuất hiện (term frequency)} của từ như là một trọng số duy nhất sẽ không đủ để phản ánh chiều sâu thông tin ẩn chứa trong mỗi văn bản.

Ở phương diện phân tích tài liệu, có những từ -- mặc dù xuất hiện ít -- lại mang nặng hàm lượng ngữ nghĩa, đóng vai trò như những \textbf{chìa khóa ngữ nghĩa (semantic cues)}, giúp phân định rõ ràng nội dung hoặc chủ đề cốt lõi của văn bản. Trái lại, những từ thường xuyên xuất hiện trong hầu hết các tài liệu -- dù mang tính phổ quát -- lại thường không đóng góp nhiều vào việc phân biệt hay nhận diện sự khác biệt giữa các nội dung. Do đó, \textbf{việc đánh giá trọng số cần được thực hiện một cách tinh tế}, trong đó phải cân nhắc đồng thời \textbf{mức độ xuất hiện trong từng tài liệu riêng biệt}, cũng như \textbf{tần suất lan tỏa của từ đó trong toàn bộ tập hợp tài liệu (corpus)}.

Một ví dụ điển hình có thể minh họa cho vấn đề này là trong một tập hợp các tài liệu chuyên ngành về công nghiệp ô-tô, từ ``ô-tô'' -- dù có xuất hiện với tần suất cao -- lại trở nên kém giá trị trong việc phân biệt các tài liệu, bởi nó mang tính nền tảng và phổ biến đến mức không còn giúp ích nhiều cho việc nhận diện sự khác biệt giữa các văn bản. Chính vì thế, \textbf{các kỹ thuật gán trọng số hiện đại như TF-IDF (Term Frequency-Inverse Document Frequency)} đã được đề xuất, nhằm điều chỉnh và hiệu chỉnh lại ảnh hưởng của những từ như vậy. Ý tưởng trung tâm của phương pháp này là \textbf{giảm trọng số của các thuật ngữ có tần suất xuất hiện cao trên toàn bộ tập tài liệu}, từ đó làm nổi bật các từ hiếm -- những thành phần mang dấu ấn cá nhân và chiều sâu chuyên biệt của từng tài liệu.

\subsubsection{Term Frequency (TF)}
Trong lĩnh vực xử lý ngôn ngữ tự nhiên và khai thác dữ liệu văn bản, \textbf{tần số thuật ngữ (Term Frequency – TF)} là một đại lượng cơ bản nhưng mang giá trị cốt lõi trong việc định lượng mức độ xuất hiện của một từ hoặc cụm từ trong một tài liệu nhất định. Khái niệm này đóng vai trò như một chỉ số phản ánh \textbf{mức độ nổi bật nội tại} của thuật ngữ trong bối cảnh của tài liệu đó, từ đó góp phần xác định tầm quan trọng tương đối của nó so với các thuật ngữ khác cùng tồn tại trong văn bản.

Về mặt hình thức, term frequency được tính bằng cách \textbf{lấy số lần xuất hiện thực tế của một từ khóa} trong tài liệu, sau đó \textbf{chia cho tổng số từ} trong tài liệu đó. Tỷ lệ này mang ý nghĩa thống kê rõ rệt: một từ xuất hiện với tần suất cao trong một văn bản cụ thể nhiều khả năng sẽ đóng vai trò then chốt trong việc phản ánh nội dung hoặc chủ đề mà tài liệu đề cập tới. Ngược lại, những từ hiếm gặp – nếu xuất hiện – sẽ chỉ chiếm một phần rất nhỏ trong tổng thể và do đó có chỉ số TF thấp hơn. \cite{manning2008introduction}

Công thức thường thấy cho TF \cite{manning2008introduction}:
\begin{equation}
    tf(t, d) = \frac{f_{t, d}}{\sum_{t'}f_{t', d}}
\end{equation}

Chính vì đặc tính tỷ lệ này, TF không chỉ đơn thuần là một con số, mà là một \textbf{dạng lượng hóa ý nghĩa ngôn ngữ}, cho phép mô hình hóa văn bản như một cấu trúc số liệu mà ở đó các đơn vị từ vựng được định vị và định lượng một cách khách quan. Trong thực tế, TF là thành phần quan trọng trong nhiều hệ thống đánh giá thông tin, đặc biệt là khi kết hợp với các cơ chế khác nhằm hiệu chỉnh ảnh hưởng lan rộng của từ trong toàn bộ tập tài liệu.

\subsubsection{Inverse Document Frequency}
Trong lĩnh vực xử lý ngôn ngữ tự nhiên và khai phá dữ liệu văn bản, \textbf{Inverse Document Frequency (IDF)} là một đại lượng thống kê mang tính nền tảng, được thiết kế để định lượng \textbf{độ hiếm tương đối} của một từ hoặc cụm từ trong toàn bộ tập hợp tài liệu. Nếu như \textbf{Term Frequency (TF)} phản ánh mức độ phổ biến của một thuật ngữ trong nội bộ một tài liệu cụ thể, thì \textbf{IDF} lại đo lường \textbf{khả năng phân biệt} của thuật ngữ đó trên bình diện toàn cục -- tức là, trong phạm vi toàn bộ tập tài liệu đang xét.

Ý tưởng trung tâm của IDF nằm ở chỗ: những từ ngữ xuất hiện trong rất nhiều tài liệu - như các từ chức năng hoặc từ phổ thông -- thường không cung cấp nhiều giá trị thông tin cho quá trình phân loại hay truy xuất. Ngược lại, một từ xuất hiện \textbf{hiếm hoi nhưng đều đặn} chỉ trong một số ít tài liệu sẽ mang hàm lượng thông tin cao hơn, vì nó có thể giúp \textbf{phân biệt rõ ràng} các chủ đề hoặc nội dung khác nhau.

Về mặt hình thức, IDF của một từ được tính bằng \textbf{logarithm} của tỉ số giữa tổng số tài liệu trong tập dữ liệu và số lượng tài liệu trong đó từ đó xuất hiện ít nhất một lần:

\begin{equation}
    \text{IDF}(t) = \log \left( \frac{N}{n} \right)
\end{equation}

trong đó:

\begin{itemize}
    \item \(N\) là tổng số tài liệu trong tập dữ liệu,
    \item \(n\) là số tài liệu mà từ \(t\) xuất hiện ít nhất một lần.
\end{itemize}

Giá trị của hàm logarit trong công thức trên không chỉ giúp làm trơn phân phối tần suất mà còn bảo đảm rằng IDF không tăng quá nhanh đối với những từ cực hiếm -- giữ cho thang đo luôn trong giới hạn kiểm soát được. Khi được kết hợp cùng TF trong công thức \textbf{TF-IDF}, IDF góp phần làm nổi bật các từ ngữ không chỉ xuất hiện thường xuyên trong một tài liệu, mà còn đồng thời \textbf{ít phổ biến trong toàn bộ tập hợp} -- từ đó nâng cao độ chính xác và sắc thái ngữ nghĩa trong các mô hình truy xuất thông tin hiện đại.

\subsubsection{Term Frequency -- Inverse Document Frequency (TF-IDF)}
\textbf{TF-IDF} là một kỹ thuật thống kê phổ biến và hiệu quả trong lĩnh vực xử lý ngôn ngữ tự nhiên và khai thác dữ liệu văn bản, nhằm đo lường mức độ \textbf{quan trọng} của một thuật ngữ đối với một tài liệu trong toàn bộ bộ sưu tập dữ liệu (corpus).

Phương pháp này kết hợp hai thành phần trọng yếu:

\begin{itemize}
    \item \textbf{Term Frequency (TF)}, biểu thị tần suất xuất hiện của một từ hoặc cụm từ trong một tài liệu cụ thể.
    \item \textbf{Inverse Document Frequency (IDF)}, phản ánh mức độ \textbf{hiếm} của từ đó trong toàn bộ tập tài liệu, qua việc sử dụng hàm logarit của tỷ lệ giữa tổng số tài liệu và số tài liệu chứa từ đó.
\end{itemize}

Công thức chung được xây dựng dưới dạng:

\begin{equation}
    \text{TF-IDF}(t, d, D) = \text{TF}(t, d) \cdot \text{IDF}(t, D)
\end{equation}

trong đó:

\begin{itemize}
    \item \(\text{TF}(t,d)\) là tỷ lệ giữa số lần từ \(t\) xuất hiện trong tài liệu \(d\) và tổng số từ trong \(d\);
    \item \(\text{IDF}(t,D) = \log\frac{N}{n_t}\), với \(N\) là tổng số tài liệu trong tập dữ liệu \(D\), và \(n_t\) là số tài liệu chứa ít nhất một lần từ \(t\).
\end{itemize}

Về mặt ngữ nghĩa, TF-IDF giúp \textbf{làm sáng} những từ khóa có tần suất cao trong từng tài liệu cụ thể (qua TF), đồng thời \textbf{hạ thấp vai trò của những từ phổ thông}, xuất hiện rộng khắp trong corpus (qua IDF). Kết quả thu được là một đại lượng trọng số, cho biết \textbf{độ quan trọng tổng hòa} của thuật ngữ đó trong ngữ cảnh của tài liệu lẫn toàn bộ tập hợp văn bản.

Nhờ khả năng cân bằng giữa tần suất nội dung và mức độ phân biệt khắp bộ dữ liệu, TF-IDF trở thành một công cụ đắc lực được ứng dụng trong các hệ thống tìm kiếm, phân loại văn bản, trích xuất đặc trưng từ dữ liệu văn bản, và nhiều ứng dụng NLP khác, giúp nâng cao độ chính xác và mức độ thẩm thấu sâu sắc của các mô hình truy xuất thông tin.
\subsubsection{Vector hóa tài liệu và truy vấn}
Một điểm cần đặc biệt lưu ý trong toàn bộ quy trình tiền xử lý và biểu diễn dữ liệu là cả tài liệu (\textit{document}) và câu truy vấn (\textit{query}) đều phải được biến đổi và ánh xạ về dạng vector theo cùng một phương pháp thống nhất. Điều này xuất phát từ yêu cầu tất yếu của tính đồng nhất trong không gian biểu diễn: nếu hai thực thể không được xử lý đồng nhất, các thuật ngữ (terms) xuất hiện trong truy vấn có thể không tương thích hoặc không trùng khớp với các thuật ngữ trong tài liệu, dẫn đến việc suy giảm nghiêm trọng hiệu quả truy xuất thông tin. Việc chuẩn hóa biểu diễn không chỉ đảm bảo sự tương thích giữa truy vấn và tài liệu, mà còn nâng cao độ chính xác trong việc tính toán độ tương đồng và xếp hạng kết quả.

Xét một ví dụ minh họa để làm rõ tầm quan trọng của việc áp dụng quy trình tiền xử lý một cách nhất quán giữa tài liệu và truy vấn. Giả sử ta có hai tài liệu và một câu truy vấn, đồng thời sử dụng một tập từ dừng (\textit{stopwords}) được rút gọn từ danh sách tiếng Anh trong thư viện NLTK.

Cụ thể, hai tài liệu được giữ nguyên mà không loại bỏ từ dừng:

\begin{itemize}
    \item \textit{doc1}: \textit{what have i done and why who s this}
    \item \textit{doc2}: \textit{what have you done}
\end{itemize}

Trong khi đó, câu truy vấn có hai phiên bản:

\begin{itemize}
    \item \textit{query1} (chưa loại bỏ stopwords): \textit{what have i done}
    \item \textit{query2} (đã loại bỏ stopwords): \textit{done}
\end{itemize}

Với \textit{query1}, hệ thống có thể xếp hạng \textit{doc1} cao hơn \textit{doc2} vì toàn bộ các term trong truy vấn đều được tìm thấy đầy đủ trong \textit{doc1}, tạo ra một mức độ tương đồng ngữ nghĩa cao hơn. Ngược lại, với \textit{query2}, khi chỉ còn lại từ khóa ``\textit{done}'', cả hai tài liệu đều chứa từ này, dẫn đến việc hệ thống đánh giá chúng ngang nhau về mặt liên quan -- làm giảm độ chính xác trong việc lựa chọn tài liệu phù hợp nhất.

Tình huống này cho thấy nếu quá trình xử lý được áp dụng không đồng đều giữa truy vấn và tài liệu -- trong trường hợp này là chỉ loại bỏ stopwords ở truy vấn -- thì hiệu quả truy xuất thông tin có thể bị suy giảm đáng kể. Do đó, việc đảm bảo một quy trình xử lý thống nhất giữa hai thực thể này là điều kiện tiên quyết để duy trì tính nhất quán và độ chính xác trong các mô hình truy xuất thông tin hiện đại.

Dưới đây là một ví dụ điển hình về việc gán trọng số TF-IDF cho một tập tài liệu và một truy vấn cụ thể bằng tiếng Anh:

Giả sử chúng ta có ba tài liệu trong corpus:

\begin{table}[H]
    \caption{Ví dụ tài liệu cần gán trọng số}
    \begin{center}
        \begin{tabularx}{0.7\linewidth}{l || X}
            \toprule
            \textbf{Tài liệu} & \textbf{Nội dung}           \\
            \midrule\midrule
            \(d_1\)           & The cat sat on the mat      \\
            \(d_2\)           & My dog and cat are the best \\
            \(d_3\)           & The locals are playing      \\
            \bottomrule
        \end{tabularx}
    \end{center}
\end{table}

Và một truy vấn đơn giản:

\begin{center}
    \textbf{``The cat''}
\end{center}

Bước đầu tiên, chúng ta biểu diễn mỗi tài liệu và truy vấn dưới dạng vector TF-IDF. Ví dụ, tính TF-IDF cho các từ trong truy vấn như sau:

\begin{itemize}
    \item \(\text{TF}(\text{``cat''}, d_{1}) = \frac{1}{6}\); \(\text{IDF}(\text{``cat''}) = \log\left(\frac{3}{2}\right) \approx 0.18\), nên \(\text{TF-IDF} \approx 0.03\)
    \item \(\text{TF}(\text{``cat''}, d_{2}) = \frac{1}{7}\); \(\text{IDF}(\text{``cat''}) = 0.18\); \(\text{TF-IDF} \approx 0.025\)
    \item \(\text{TF}(\text{``cat''}, d_{3}) = 0\); nên \(\text{TF-IDF} = 0\)
\end{itemize}

Trong khi đó, từ ``the'' có IDF = \(\log(\frac{3}{3})\) = 0, dẫn đến TF-IDF luôn bằng 0, bất kể tài liệu nào chứa từ này.

Sau khi tính toán, chúng ta được hai vector TF-IDF:

\begin{table}[H]
    \caption{Giá trị TF-IDF lập theo term cho từng tài liệu}
    \begin{center}
        \begin{tabularx}{\linewidth}{l || X X X l}
            \toprule
            \textbf{Term} & \textbf{\(d_1\) TF-IDF} & \textbf{\(d_2\) TF-IDF} & \textbf{\(d_3\) TF-IDF} & \textbf{Query TF-IDF} \\
            \midrule\midrule
            \textbf{the}  & 0                       & 0                       & 0                       & 0                     \\
            \textbf{cat}  & 0.0306                  & 0.0252                  & 0                       & 0.18                  \\
            \bottomrule
        \end{tabularx}
    \end{center}
\end{table}

Tiếp theo, biểu diễn truy vấn ``The cat'' cũng dưới dạng TF-IDF vector — ở đây, ``cat'' nhận trọng số \~0.18, còn ``the'' vẫn là 0.

Cuối cùng, hệ thống tính \textbf{độ tương đồng cosine} giữa vector truy vấn và từng vector tài liệu, rồi xếp hạng dựa trên giá trị đó:

\begin{enumerate}
    \item \(d_{1}\) -- có độ tương đồng cao nhất (vì chứa đầy đủ cả hai term),
    \item \(d_{2}\) -- là lựa chọn thứ hai,
    \item \(d_{3}\) -- đứng cuối cùng.
\end{enumerate}

Thông qua ví dụ này, chúng ta thấy rõ cách TF-IDF không chỉ phản ánh sự hiện diện của từ trong tài liệu, mà còn điều chỉnh theo mức độ phân bố của từ đó trong toàn bộ tập dữ liệu. Kết quả là, những thuật ngữ đặc trưng hơn cho một chủ đề sẽ được hệ thống ưu tiên đánh trọng số -- và từ đó, truy vấn ``The cat'' cũng dẫn đến kết quả chính xác và có ý nghĩa hơn.

\subsubsection{Chuẩn hóa trọng số}
Chuẩn hóa trọng số, hay còn gọi là \textit{normalize}, là một bước tinh tế nhưng thiết yếu trong tiến trình biểu diễn văn bản dưới dạng vector trọng số, nhằm đảm bảo rằng mọi tài liệu, bất kể độ dài, đều được định lượng một cách công bằng và đồng nhất trong không gian truy xuất. Khi biểu diễn các tài liệu hoặc truy vấn bằng vector, nếu không có bước chuẩn hóa, các tài liệu dài với nhiều lần lặp lại của các term sẽ có xu hướng chiếm ưu thế không công bằng so với các tài liệu ngắn hơn, mặc dù có thể mức độ liên quan của chúng lại không thực sự cao hơn. Bằng cách đưa các trọng số về cùng một miền giá trị chuẩn hóa -- thường là đoạn \([0, 1]\) -- ta không những làm giảm ảnh hưởng không mong muốn của độ dài tài liệu, mà còn biến đổi các vector thành các đại diện đơn vị, có độ dài bằng 1 và cùng phương hướng trong không gian vector chuẩn. Điều này tạo điều kiện thuận lợi cho việc áp dụng các phép đo như cosine similarity, vì mọi phép đo khi đó chỉ còn phụ thuộc vào hướng của vector -- tức là nội dung ngữ nghĩa -- thay vì độ lớn của nó. Về mặt kỹ thuật, một trong những công thức chuẩn hóa trọng số thường dùng là:

\begin{equation}
    w_{ik} = \frac{tf_{ik} \cdot idf_{ik}}{\sqrt{\sum_{k=1}^{N} (tf_{ik} \cdot idf_{ik})^2}}
\end{equation}

trong đó \(w_{ik}\) là trọng số chuẩn hóa của term \(k\) trong tài liệu \(i\), đảm bảo rằng tổng bình phương các trọng số của mọi term trong một document sẽ luôn bằng 1. Ngoài ra, trong thực tiễn triển khai, người ta còn sử dụng những phương pháp chuẩn hóa khác như hệ thống S.M.A.R.T., trong đó trọng số được xác định theo công thức:

\begin{equation}
    w_{kd} = collect_k \cdot fred_{kd} \cdot \frac{1}{norm}
\end{equation}

với mỗi biến số phản ánh các yếu tố đặc trưng trong mối quan hệ giữa term, tài liệu và tập hợp tài liệu. Những phương pháp như vậy giúp điều chỉnh độ quan trọng của từ khóa không chỉ dựa vào tần suất xuất hiện, mà còn kết hợp với các yếu tố về ngữ cảnh và phân bố để nâng cao tính chính xác và công bằng trong hệ thống truy xuất thông tin.

