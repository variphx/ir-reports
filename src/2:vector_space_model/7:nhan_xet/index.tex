\section{Nhận xét}
Dưới ánh sáng của quá trình khảo cứu toàn diện về mô hình Không gian Vector (\textit{Vector Space Model}), ta có thể đi đến một cái nhìn tổng thể về những khả năng cốt lõi mà mô hình này mang lại cho nhiệm vụ truy vấn thông tin trong các hệ thống hiện đại. Trước hết, mô hình cho phép thực hiện việc \textbf{xếp hạng các kết quả truy vấn} một cách linh hoạt và định lượng -- nghĩa là không chỉ xác định xem một tài liệu có liên quan hay không, mà còn cho phép đo lường mức độ liên quan ấy để sắp xếp tài liệu theo trật tự ưu tiên. Đây là một điểm mạnh nổi bật so với các mô hình nhị phân cổ điển.

Mặt khác, cả \textbf{tài liệu và câu truy vấn đều được mô hình hóa dưới dạng các vector số thực} trong một không gian nhiều chiều, trong đó mỗi chiều đại diện cho một \textit{term} (thuật ngữ chỉ mục) trong tập từ vựng đã được chuẩn hóa và thống nhất. Việc biểu diễn như vậy không chỉ giúp tiêu chuẩn hóa cách thức hệ thống nhìn nhận các thực thể ngôn ngữ, mà còn mở đường cho các phép toán đại số tuyến tính có thể được áp dụng một cách hiệu quả để phân tích mối quan hệ giữa các văn bản.

\textbf{Mỗi chiều của vector đại diện cho một chỉ mục cụ thể}, và giá trị trên chiều ấy -- hay còn gọi là trọng số -- là một số thực phản ánh \textbf{mức độ quan trọng của thuật ngữ đó trong ngữ cảnh của văn bản hoặc câu truy vấn}. Những trọng số này, thường được tính bằng các công thức như TF, IDF hay TF-IDF, là nền tảng để tạo ra các biểu diễn giàu thông tin.

Từ những biểu diễn này, mô hình sử dụng \textbf{độ đo cosine} để xác định \textbf{độ tương đồng giữa hai vector}, chính là giữa truy vấn và tài liệu. Phép đo này phản ánh mức độ định hướng tương đồng giữa hai thực thể trong không gian ngữ nghĩa, cho phép suy ra mức độ liên quan một cách mượt mà và hình học.

Không dừng lại ở đó, vì mọi thực thể đều được quy đổi về dạng vector -- dù là tài liệu, truy vấn, hay thậm chí là một câu văn đơn lẻ -- nên \textbf{mọi vector trong hệ thống đều có thể được đem ra so sánh trực tiếp với nhau về mặt ngữ nghĩa}. Điều này mở ra một không gian thao tác phong phú, nơi các thực thể ngôn ngữ được phân tích như các điểm hình học, giúp hệ thống truy xuất vận hành theo lối tư duy toán học chính xác và nhất quán.

Qua toàn bộ tiến trình nghiên cứu, có thể thấy rõ rằng mô hình Không gian Vector không chỉ đơn thuần là một kỹ thuật biểu diễn, mà còn là một triết lý mô hình hóa ngôn ngữ -- nơi văn bản được hiểu như những vector sống động, sẵn sàng được so sánh, xếp hạng và suy diễn trong một vũ trụ hình học đầy tiềm năng. Những ưu điểm và hạn chế cụ thể của mô hình sẽ được trình bày chi tiết hơn ở phần sau, nhằm tiếp tục làm rõ vị thế của mô hình này trong hệ sinh thái truy xuất thông tin hiện đại

\subsection{Ưu điểm}
Mô hình Không gian Vector (\textit{Vector Space Model}) sở hữu nhiều ưu điểm nổi bật, khiến nó trở thành một trong những phương pháp được chấp nhận rộng rãi và ứng dụng phổ biến trong lĩnh vực truy xuất thông tin. Trước hết, đây là một mô hình đã được nghiên cứu sâu rộng trong suốt nhiều thập kỷ, từ đó hình thành nên một nền tảng lý thuyết vững chắc và được củng cố qua vô số thực nghiệm. Tính lịch sử này không chỉ mang lại độ tin cậy cao mà còn tạo điều kiện thuận lợi cho việc tiếp cận, học tập và áp dụng mô hình trong thực tiễn.

Bên cạnh đó, mô hình cũng được đánh giá cao nhờ vào \textbf{cấu trúc đơn giản, dễ hiểu và dễ triển khai}, giúp người thực hiện có thể nhanh chóng hiện thực hóa các nguyên lý lý thuyết thành hệ thống hoạt động hiệu quả. Một điểm mạnh then chốt khác là khả năng \textbf{xếp hạng các tài liệu truy xuất được dựa trên mức độ liên quan định lượng} với truy vấn đầu vào -- một khả năng vượt trội so với các mô hình nhị phân vốn chỉ cho phép phân loại rạch ròi tài liệu là liên quan hoặc không liên quan.

Cuối cùng, \textbf{tính linh hoạt trong việc gán trọng số cho các thuật ngữ} cũng là một ưu điểm lớn của mô hình này. Với nhiều phương pháp như TF, IDF hay TF-IDF, người thiết kế hệ thống có thể tùy chỉnh các cách đánh giá tầm quan trọng của từ vựng trong tài liệu và truy vấn, từ đó nâng cao chất lượng và độ chính xác của kết quả truy xuất. Chính nhờ sự kết hợp hài hòa giữa tính chặt chẽ về lý thuyết và sự mềm dẻo trong thực tiễn, mô hình Không gian Vector đã và đang giữ một vai trò then chốt trong nhiều hệ thống truy xuất thông tin hiện đại

\subsection{Nhược điểm}
Bên cạnh những ưu điểm vượt trội, mô hình Không gian Vector (\textit{Vector Space Model}) cũng bộc lộ một số hạn chế cố hữu, xuất phát từ chính bản chất hình học giản lược của nó trong việc biểu diễn ngôn ngữ tự nhiên. Trước hết, mô hình này \textbf{bỏ qua hoàn toàn thứ tự xuất hiện của các từ khóa} trong tài liệu và truy vấn. Điều đó đồng nghĩa với việc mọi khía cạnh liên quan đến ngữ pháp, cú pháp hay cấu trúc của câu -- vốn có vai trò quan trọng trong việc truyền tải nghĩa -- đều không được phản ánh trong không gian biểu diễn.

Thêm vào đó, \textbf{mỗi chiều trong không gian vector tương ứng với một từ khóa đơn lẻ}, song không hề có bất kỳ sự ràng buộc hay liên hệ nào về mặt ngữ nghĩa giữa các chiều đó. Điều này khiến mô hình không thể nắm bắt được các mối quan hệ ngữ nghĩa sâu sắc giữa từ ngữ, chẳng hạn như đồng nghĩa, trái nghĩa hay tính chất phân cấp trong từ vựng. Không gian mà các vector tồn tại cũng chỉ được giới hạn \textbf{trong phần dương}, do các trọng số như TF hay TF-IDF không thể nhận giá trị âm. Sự hạn chế này làm giảm tính biểu cảm của mô hình trong những tình huống cần biểu diễn mối quan hệ phủ định hoặc đối lập giữa các khái niệm.

Một trong những điểm yếu then chốt của mô hình là \textbf{cơ chế so khớp từ khóa cứng nhắc}. Nếu truy vấn và tài liệu không có bất kỳ từ khóa chung nào -- dù nội dung thực tế có thể tương đồng về mặt ý tưởng -- thì \textbf{độ tương đồng được tính toán sẽ bằng không tuyệt đối}, phản ánh một khoảng cách triệt để không thực sự hợp lý trong nhiều trường hợp ngữ nghĩa. Chính vì những hạn chế này, mô hình Không gian Vector tuy mạnh mẽ trong các tình huống đơn giản, nhưng cũng cần được bổ sung hoặc cải tiến khi áp dụng cho các bài toán truy xuất thông tin phức tạp và giàu ngữ nghĩa hơn
