\subsection{Nhược điểm}
Bên cạnh những ưu điểm vượt trội, mô hình Không gian Vector (\textit{Vector Space Model}) cũng bộc lộ một số hạn chế cố hữu, xuất phát từ chính bản chất hình học giản lược của nó trong việc biểu diễn ngôn ngữ tự nhiên. Trước hết, mô hình này \textbf{bỏ qua hoàn toàn thứ tự xuất hiện của các từ khóa} trong tài liệu và truy vấn. Điều đó đồng nghĩa với việc mọi khía cạnh liên quan đến ngữ pháp, cú pháp hay cấu trúc của câu -- vốn có vai trò quan trọng trong việc truyền tải nghĩa -- đều không được phản ánh trong không gian biểu diễn.

Thêm vào đó, \textbf{mỗi chiều trong không gian vector tương ứng với một từ khóa đơn lẻ}, song không hề có bất kỳ sự ràng buộc hay liên hệ nào về mặt ngữ nghĩa giữa các chiều đó. Điều này khiến mô hình không thể nắm bắt được các mối quan hệ ngữ nghĩa sâu sắc giữa từ ngữ, chẳng hạn như đồng nghĩa, trái nghĩa hay tính chất phân cấp trong từ vựng. Không gian mà các vector tồn tại cũng chỉ được giới hạn \textbf{trong phần dương}, do các trọng số như TF hay TF-IDF không thể nhận giá trị âm. Sự hạn chế này làm giảm tính biểu cảm của mô hình trong những tình huống cần biểu diễn mối quan hệ phủ định hoặc đối lập giữa các khái niệm.

Một trong những điểm yếu then chốt của mô hình là \textbf{cơ chế so khớp từ khóa cứng nhắc}. Nếu truy vấn và tài liệu không có bất kỳ từ khóa chung nào -- dù nội dung thực tế có thể tương đồng về mặt ý tưởng -- thì \textbf{độ tương đồng được tính toán sẽ bằng không tuyệt đối}, phản ánh một khoảng cách triệt để không thực sự hợp lý trong nhiều trường hợp ngữ nghĩa. Chính vì những hạn chế này, mô hình Không gian Vector tuy mạnh mẽ trong các tình huống đơn giản, nhưng cũng cần được bổ sung hoặc cải tiến khi áp dụng cho các bài toán truy xuất thông tin phức tạp và giàu ngữ nghĩa hơn
