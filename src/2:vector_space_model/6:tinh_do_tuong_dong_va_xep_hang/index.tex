\section{Tính độ tương đồng và xếp hạng}
Khi các tài liệu (documents) và câu truy vấn (queries) đã được ánh xạ thành các vector trọng số trong không gian |V| chiều, nơi |V| đại diện cho kích thước của từ vựng \allowbreak (vocabulary) hay chính là số lượng các thuật ngữ (terms) riêng biệt, mỗi vector trong không gian này không đơn thuần chỉ là một chuỗi số liệu trừu tượng mà là một điểm mang giá trị hình học cụ thể, có vị trí, có hướng, và có độ lớn. Mỗi điểm được định danh bởi tập hợp các trọng số tương ứng với các term trong V, tuy nhiên, để tạo thành một vector hoàn chỉnh -- một đối tượng có phương và độ dài -- thì ta cần thêm một điểm mốc để định hướng chuyển động. Điểm mốc đó không gì khác chính là gốc tọa độ \(\mathbf{0} = (0, 0, \dots, 0)\) trong không gian |V| chiều, đóng vai trò như điểm khởi đầu của mọi vector biểu diễn document hoặc query. Nhờ có điểm gốc này, mỗi document hoặc query được hình dung như một vector kéo dài từ điểm \(\mathbf{0}\) đến vị trí tương ứng với tập trọng số của chính nó trong không gian.

Sau khi đã hoàn tất giai đoạn gán trọng số cho từng thành phần trong các vector tài liệu và truy vấn -- tức là sau khi đã xây dựng được biểu diễn đại số của chúng -- bài toán tiếp theo cần giải quyết là xác định mức độ liên quan (relevance) giữa mỗi tài liệu và truy vấn. Vấn đề đặt ra ở đây là: làm thế nào để xếp hạng các tài liệu trong không gian |V| chiều theo độ phù hợp với truy vấn đã cho? Câu trả lời nằm chính trong cách thức ta biểu diễn các tài liệu và truy vấn như những vector trong cùng một không gian hình học. Bởi vì tất cả các vector đều được định nghĩa cùng trong một hệ tọa độ và cùng dựa trên cùng một từ vựng, ta có thể tiến hành so sánh sự gần nhau -- hay còn gọi là \textit{proximity} -- giữa chúng bằng các phương pháp định lượng. Trong ngữ cảnh này, proximity chính là một dạng của similarity (độ tương đồng), phản ánh qua khoảng cách hình học hoặc góc giữa các vector. Sự tương đồng càng cao, tức là vector truy vấn càng gần với vector tài liệu trong không gian |V| chiều, thì mức độ liên quan của tài liệu đó đối với truy vấn càng lớn. Từ đây, việc xếp hạng các tài liệu trở thành một quá trình đo đạc và sắp xếp độ gần giữa các vector tài liệu và vector truy vấn, dựa trên các hàm đo cụ thể như cosine similarity hay các chuẩn hình học khác. Bước kế tiếp trong tiến trình truy xuất chính là định lượng sự gần gũi ấy một cách chính xác, để xây dựng nên một thứ tự truy xuất có ý nghĩa về mặt ngữ nghĩa và hình học

\subsection{Độ tương đồng cosine}
Độ đo tương đồng cosine, hay còn gọi là \textit{cosine similarity}, là một phương pháp hình học được áp dụng phổ biến trong các mô hình không gian vector nhằm định lượng mức độ tương đồng giữa hai vector trong một không gian nhiều chiều. Trong lĩnh vực xử lý ngôn ngữ tự nhiên, phương pháp này đặc biệt hữu ích trong việc đánh giá mức độ tương quan về ngữ nghĩa giữa hai văn bản, nhờ khả năng phản ánh định hướng của các vector hơn là độ lớn tuyệt đối của chúng.

Cosine similarity được định nghĩa dựa trên góc giữa hai vector trong không gian: khi hai vector cùng hướng, góc giữa chúng nhỏ, và độ đo cosine tiến gần đến 1; khi hai vector vuông góc, góc là 90 độ, cosine bằng 0, cho thấy sự không tương đồng; và nếu chúng ngược hướng, cosine đạt giá trị -1, phản ánh một mức độ tương phản hoàn toàn. Công thức tính cosine similarity giữa hai vector A và B được viết dưới dạng:

\begin{equation}
\text{cosine\_similarity}(A, B) = \frac{A \cdot B}{\|A\| \times \|B\|}
\end{equation}

trong đó \(A \cdot B\) là tích vô hướng giữa hai vector, còn \(|A|\) và \(|B|\) là độ dài (chuẩn Euclid) của hai vector tương ứng. Giá trị kết quả luôn nằm trong đoạn \([-1, 1]\), phản ánh trực quan mối quan hệ định hướng giữa hai thực thể trong không gian biểu diễn.

Chính vì khả năng bỏ qua sự khác biệt về độ dài và tập trung vào hướng chung của các vector, cosine similarity đã trở thành một công cụ trọng yếu trong các bài toán như tìm kiếm thông tin, phân loại văn bản, gợi ý nội dung và các ứng dụng khác trong khai phá dữ liệu văn bản.
