\section{Giới thiệu}
Mô hình không gian vector (\textit{Vector Space Model} -- VSM) \cite{singh2022vectorspacemodel} là một trong những nền tảng toán học được áp dụng rộng rãi và có ảnh hưởng sâu sắc trong lĩnh vực truy xuất thông tin hiện đại. Được xem như cầu nối giữa ngôn ngữ tự nhiên và thế giới hình học trừu tượng, mô hình này mang đến một cách tiếp cận mang tính đại số, cho phép biểu diễn các tài liệu và truy vấn dưới dạng các vector số trong một không gian nhiều chiều -- nơi mỗi chiều tương ứng với một thuật ngữ trong tập chỉ mục.

Ra đời từ hệ thống truy xuất thông tin SMART \cite{buckley_salton_allan1993smart} -- một trong những hệ thống tiên phong trong lĩnh vực này -- mô hình không gian vector đã chứng minh được tính ứng dụng rộng rãi của mình trong các bài toán truy vấn, lọc thông tin, lập chỉ mục và đặc biệt là xếp hạng độ phù hợp giữa truy vấn và tài liệu.

Trong khuôn khổ của mô hình này, mỗi tài liệu cũng như mỗi câu truy vấn đều được ánh xạ thành một vector, mà thành phần của vector chính là các trọng số đại diện cho mức độ quan trọng của từng thuật ngữ trong văn bản. Những trọng số này thường được tính toán dựa trên tần suất xuất hiện của từ khóa trong tài liệu (term frequency -- \textit{TF}), được điều chỉnh bởi mức độ phổ biến của từ đó trong toàn bộ tập hợp tài liệu (inverse document frequency -- \textit{IDF}), tạo thành công thức trọng số nổi tiếng TF-IDF. Trọng số càng cao, từ khóa càng mang tính đặc trưng và có khả năng phân biệt tài liệu đó với các tài liệu khác.

Khi các tài liệu và truy vấn đã được mã hóa dưới dạng các vector trong cùng một không gian, bài toán xác định mức độ liên quan giữa chúng trở thành một phép toán hình học: khoảng cách -- hay chính xác hơn, góc -- giữa hai vector. Phép đo thường được sử dụng nhất là \textit{độ đo cosine} (cosine similarity), phản ánh mức độ ``hướng cùng chiều'' giữa truy vấn và tài liệu trong không gian đa chiều. Giá trị cosine càng lớn thì mức độ tương đồng càng cao, đồng nghĩa với việc tài liệu đó càng phù hợp với nội dung truy vấn.

Thông qua cơ chế xếp hạng dựa trên độ tương đồng, mô hình không gian vector cho phép hệ thống truy xuất đưa ra một danh sách các tài liệu sắp xếp theo thứ tự giảm dần của mức độ liên quan -- từ đó mang lại cho người dùng những kết quả tìm kiếm chính xác và có ý nghĩa nhất. Không dừng lại ở việc xử lý ngôn ngữ dưới dạng dữ liệu, mô hình này còn mở ra một góc nhìn đầy hình tượng: nơi những dòng chữ được chuyển hóa thành hình khối toán học, và hành trình đi tìm tri thức trở thành một phép đo khoảng cách giữa hai điểm trong không gian tư duy.
