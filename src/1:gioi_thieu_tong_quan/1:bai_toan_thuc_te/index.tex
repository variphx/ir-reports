\section{Bài toán thực tế}
Trong bối cảnh dữ liệu số ngày càng gia tăng nhanh chóng, nhu cầu tìm kiếm và truy xuất thông tin một cách hiệu quả trở nên thiết yếu hơn bao giờ hết. Từ thực tiễn này, lĩnh vực Truy xuất Thông tin (Information Retrieval, về sau sẽ gọi là IR) đã hình thành và phát triển như một nhánh quan trọng của khoa học máy tính. Thay vì đơn thuần xử lý dữ liệu, truy xuất thông tin tập trung vào việc xác định, đánh giá và cung cấp những tài liệu liên quan đến nhu cầu của người dùng từ một kho dữ liệu lớn. Quá trình này bao gồm các bước như lập chỉ mục, truy vấn và xếp hạng kết quả, với sự hỗ trợ của nhiều thuật toán và mô hình tính toán hiện đại.

Thông qua việc mô hình hóa mối quan hệ giữa truy vấn và tài liệu, truy xuất thông tin cho phép cải thiện hiệu quả tiếp cận tri thức trong nhiều bối cảnh, từ công cụ tìm kiếm trên Internet, hệ thống đề xuất nội dung, đến các thư viện số và kho học liệu điện tử. Đặc biệt, với sự phát triển của trí tuệ nhân tạo và học máy, các hệ thống truy hồi ngày nay không chỉ dựa vào từ khóa đơn lẻ mà còn có khả năng hiểu ngữ nghĩa, phân tích ngữ cảnh và thậm chí dự đoán nhu cầu thông tin tiềm ẩn của người dùng. Điều này góp phần nâng cao trải nghiệm truy xuất và đảm bảo thông tin được cung cấp phù hợp hơn với mục đích sử dụng.

Bên cạnh đó, IR còn đóng vai trò quan trọng trong khai phá dữ liệu, xử lý ngôn ngữ tự nhiên và phân tích dữ liệu lớn. Tính ứng dụng rộng rãi cùng khả năng thích nghi với sự biến động của công nghệ khiến lĩnh vực này trở thành một nền tảng không thể thiếu trong hệ sinh thái tri thức hiện đại.