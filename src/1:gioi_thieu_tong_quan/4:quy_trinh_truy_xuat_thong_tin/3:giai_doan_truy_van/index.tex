\subsection{Giai đoạn truy vấn}
\subsubsection{Xử lý truy vấn}
Khi người dùng phát sinh nhu cầu thông tin, hệ thống truy xuất sẽ tiếp nhận biểu hiện cụ thể của nhu cầu ấy thông qua một \textbf{câu truy vấn} (\textit{query}). Câu truy vấn không chỉ đơn thuần là chuỗi ký tự, mà chính là hình ảnh nén lại của một mục tiêu tri thức -- được hệ thống tiếp nhận như một tín hiệu định hướng để bắt đầu quá trình tìm kiếm. Giống như cách mà các tài liệu đã được xử lý trước đó, truy vấn cũng trải qua các bước tiền xử lý như chuẩn hóa, tách từ, loại bỏ từ dừng và chuyển hóa về dạng ngữ nghĩa thích hợp, nhằm tạo nên sự tương thích cấu trúc giữa truy vấn và kho dữ liệu lưu trữ.

\subsubsection{Truy xuất chỉ mục}
Sau khi truy vấn đã được chuẩn hóa, từng \textit{term} (từ khóa) được sinh ra từ câu truy vấn sẽ đóng vai trò như những chìa khóa tìm kiếm, được đưa vào hệ thống để đối chiếu với tập \textit{chỉ mục} đã được xây dựng. Đây là giai đoạn trung tâm của toàn bộ tiến trình truy xuất -- nơi diễn ra sự tương tác giữa tri thức cần tìm và tri thức đang có. Quá trình này là một phép so sánh tinh vi giữa các từ khóa trong truy vấn và các từ khóa đã được lập chỉ mục trong các tài liệu, nhằm xác định xem tài liệu nào chứa đựng những nội dung có khả năng thỏa mãn nhu cầu thông tin.

\subsubsection{Xếp hạng kết quả}
Khi các tài liệu phù hợp được xác định, hệ thống sẽ không trả về kết quả một cách ngẫu nhiên hay theo thứ tự xuất hiện, mà thực hiện một bước sắp xếp có ý thức -- gọi là \textbf{xếp hạng kết quả}. Mỗi tài liệu được gán một điểm số phản ánh mức độ liên quan giữa nội dung của nó và truy vấn đã đưa vào. Mức điểm này được tính toán dựa trên công thức riêng của từng mô hình truy xuất thông tin, như TF-IDF, BM25 hay các phương pháp học sâu hiện đại. Kết quả cuối cùng là một danh sách các tài liệu được sắp xếp theo thứ tự giảm dần của độ liên quan -- một hành lang tinh lọc mà ở đó, tri thức gần nhất với nhu cầu người dùng sẽ được đặt lên hàng đầu, dẫn lối cho hành trình tìm kiếm tri thức trở nên rõ ràng và hiệu quả hơn bao giờ hết.
