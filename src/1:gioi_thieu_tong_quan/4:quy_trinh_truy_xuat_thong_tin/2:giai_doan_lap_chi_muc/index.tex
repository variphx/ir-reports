\subsection{Giai đoạn lập chỉ mục}
\textbf{Lập chỉ mục} (\textit{Indexing}) là một trong những trụ cột cấu thành nên hiệu quả hoạt động của hệ thống \textbf{truy xuất thông tin (Information Retrieval -- IR)}. Trong bản chất của mình, đây không chỉ đơn thuần là một thao tác kỹ thuật, mà còn là quá trình thiết kế nên một cấu trúc logic tinh tế -- nơi mà các yếu tố cốt lõi của ngôn ngữ được mã hóa, tổ chức và lưu trữ để phục vụ cho quá trình tìm kiếm diễn ra nhanh chóng và chính xác.

Quá trình lập chỉ mục khởi đầu bằng một bước tiền xử lý cẩn trọng. Tại đây, các văn bản thô sẽ được thanh lọc và chuẩn hóa thông qua các thao tác như loại bỏ \textbf{stopwords}, chuyển về chữ thường, xử lý dấu câu, tách từ và lấy gốc từ (stemming/lemmatization). Mục tiêu của bước này là loại bỏ đi những yếu tố gây nhiễu, đồng thời làm sáng tỏ những đơn vị thông tin có giá trị -- vốn là những \textit{term} (thuật ngữ chỉ mục) sẽ trở thành linh hồn của bộ chỉ mục sau này.

Khi văn bản đã được tiền xử lý, bước tiếp theo là \textbf{phân tích văn bản} nhằm xác định các thuật ngữ đại diện. Các \textit{term} này được trích xuất như những đại biểu mang tính ngữ nghĩa cho nội dung văn bản, và từ đó trở thành chìa khóa để kết nối câu truy vấn của người dùng với các tài liệu phù hợp trong kho dữ liệu.

Tiếp theo là công đoạn \textbf{xây dựng bộ chỉ mục} -- một cấu trúc dữ liệu phức hợp nhưng tối ưu, trong đó mỗi thuật ngữ sẽ được ánh xạ đến danh sách các tài liệu chứa nó, đi kèm với các thuộc tính bổ trợ như \textbf{vị trí xuất hiện}, \textbf{tần suất}, hoặc các chỉ số đánh giá mức độ quan trọng của thuật ngữ trong từng tài liệu cụ thể. Chính nhờ vào những chỉ số này mà hệ thống có thể thực hiện việc xếp hạng các tài liệu theo mức độ liên quan trong giai đoạn truy vấn.

Sau khi hoàn tất, \textbf{bộ chỉ mục được lưu trữ} như một bản đồ thu gọn của kho tri thức, đóng vai trò là nơi tham chiếu cho tất cả các hoạt động truy xuất sau này. Việc truy vấn không còn phải thực hiện trực tiếp trên toàn bộ văn bản -- điều vốn tiêu tốn thời gian và tài nguyên -- mà được tối ưu hóa thông qua việc dò tìm trên bộ chỉ mục đã được tổ chức chặt chẽ và hiệu quả.

Tựu trung, \textbf{quá trình lập chỉ mục không chỉ là một bước trung gian}, mà là một tầng nền tảng quan trọng trong toàn bộ hệ sinh thái truy xuất thông tin. Một bộ chỉ mục được thiết kế tốt sẽ không chỉ nâng cao \textbf{độ chính xác} và \textbf{tốc độ truy vấn}, mà còn góp phần mở rộng \textbf{khả năng bao phủ thông tin}, mang đến cho người dùng những kết quả tìm kiếm đầy đủ, có ý nghĩa và kịp thời. Trong thời đại dữ liệu bùng nổ, khi mà việc tìm đúng thông tin đúng lúc trở nên sống còn, thì vai trò của lập chỉ mục càng trở nên không thể thay thế -- như một hệ thần kinh kết nối kho tàng tri thức với nhu cầu tức thời của con người
